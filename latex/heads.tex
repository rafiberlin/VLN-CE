\begin{table}
\centering
\caption{\label{tab:heads}Influence of the GPT Heads. Subset of experiments' results, grouped by agent and ranked by descending SPL on the Validation Unseen data split. The rank in column \# is also used as a look up id in table \ref{tab:all-configs-final} to link the corresponding training configuration.     \newline The agents are based on Decision Transformer ('DT'), Enhanced Decision Transformer ('E-DT') or Full Decision Transformer ('F-DT').}
\begin{tabular}{@{\hskip3pt}c@{\hskip3pt}c@{\hskip3pt}c@{\hskip3pt}c@{\hskip3pt}c@{\hskip3pt}c@{\hskip3pt}c@{\hskip3pt}c@{\hskip3pt}c@{\hskip3pt}c@{\hskip3pt}c@{\hskip3pt}c@{\hskip3pt}c@{\hskip3pt}c@{\hskip3pt}c}
\toprule
\textbf{\#} & \textbf{Agent} & \textbf{GPT Heads} & \multicolumn{6}{c}{\textbf{Val Seen}} & \multicolumn{6}{c}{\textbf{Val Unseen}} \\
\cmidrule(l){4-9} \cmidrule(l){10-15} \textbf{~} &     \textbf{~} &         \textbf{~} &       \textbf{TL} & \textbf{NE} & \textbf{nDTW} & \textbf{OS} & \textbf{SR} & \textbf{SPL} &         \textbf{TL} & \textbf{NE} & \textbf{nDTW} & \textbf{OS} & \textbf{SR} & \textbf{SPL} \\
\midrule
         34 &             DT &                 16 &             8.447 &       8.481 &         0.458 &       0.356 &       0.240 &        0.226 &               7.972 &       9.480 &         0.401 &       0.245 &       0.170 &        0.156 \\
         77 &           E-DT &                  8 &             8.359 &       8.913 &         0.440 &       0.312 &       0.212 &        0.200 &               8.065 &       9.644 &         0.395 &       0.228 &       0.152 &        0.140 \\
         86 &           E-DT &                 16 &             8.574 &       8.907 &         0.437 &       0.329 &       0.243 &        0.227 &               8.019 &       9.160 &         0.406 &       0.247 &       0.153 &        0.139 \\
         93 &           F-DT &                 16 &             8.460 &       8.335 &         0.473 &       0.343 &       0.239 &        0.222 &               7.682 &       9.266 &         0.406 &       0.245 &       0.150 &        0.137 \\
         99 &           F-DT &                  8 &             8.395 &       8.065 &         0.476 &       0.348 &       0.248 &        0.232 &               7.801 &       9.390 &         0.397 &       0.222 &       0.150 &        0.136 \\
\bottomrule
\end{tabular}
\end{table}
